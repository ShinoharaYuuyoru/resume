% !TEX TS-program = xelatex
% !TEX encoding = UTF-8 Unicode
% !Mode:: "TeX:UTF-8"

\documentclass{resume}
\usepackage{zh_CN-Adobefonts_external} % Simplified Chinese Support using external fonts (./fonts/zh_CN-Adobe/)
%\usepackage{zh_CN-Adobefonts_internal} % Simplified Chinese Support using system fonts
\usepackage{linespacing_fix} % disable extra space before next section
\usepackage{cite}

\begin{document}
\pagenumbering{gobble} % suppress displaying page number

\name{Quanyu PIAO}

% {E-mail}{mobilephone}{homepage}
% be careful of _ in emaill address
\contactInfo{(+86)15624951831}{quanyu.piao@gmail.com}{}{}
% {E-mail}{mobilephone}
% keep the last empty braces!
%\contactInfo{xxx@yuanbin.me}{(+86) 131-221-87xxx}{}
 
% \section{个人总结}
% 本人在校成绩优秀、乐观向上,工作负责、自我驱动力强、热爱尝试新事物,认同开放、连接、共享的Web在未来的不可替代性。在校期间长期从事可视分析(Web的2D/3D时空可视化)相关研究,对Web技术发展趋势及前端工程化解决方案有浓厚兴趣。\textbf{现任职于阿里巴巴集团。}

% \section{\faGraduationCap\ 教育背景}
\section{Education}
\datedsubsection{\textbf{Waseda University},Computer Science \& Communication,\textit{Master}}{2019.04 - 2021.03}
\datedsubsection{\textbf{Beijing Institute of Technology},Computer Science \& Technology,\textit{Bachelor}}{2014.09 - 2018.06}
% \section{\faCogs\ IT 技能}
\section{Skills}
% increase linespacing [parsep=0.5ex]
\begin{itemize}
  \item \textbf{Programming Language}: C, C++, Python, Java(Android), Swift(iOS)
  \item \textbf{Framework}: Doris(Apache); PyTorch, TensorFlow 1.x; OpenCV
  \item \textbf{Keyword}: Data Warehouse / Machine Learning / Client Development
\end{itemize}

% \end{itemize}

\section{Internship}
\datedsubsection{\textbf{SOFTBRAIN Co., Ltd.}, iOS Engineer}{2019.08 - 2020.05}
\begin{itemize}
  \item Develop a business card detecting module with OpenCV in iOS, 1 teammate.
  \item Transfer a part of the existing Objective-C code to Swift code, independently.
  % \item 小组(2人)负责开发并优化iOS客户端名片识别模块(OpenCV)。
  % \item 独立负责一部分原有的Objective-C项目向Swift的迁移。
\end{itemize}

\datedsubsection{\textbf{Amazon Japan}, Summer Internship}{2019.09}
\begin{itemize}
  \item Experience developing Amazon Alexa Skills.
  % \item 体验Amazon Alexa Skills的开发。
\end{itemize}

\datedsubsection{\textbf{SOHU, Inc.}, Data Warehouse Engineer}{2018.12 - 2019.03}
\begin{itemize}
  \item Transfer existing data to Doris, including stability and performance testing, load balance, etc.
  \item Learn the Kerberos protocol; Establish the Grafana server monitoring platform.
  % \item 参与向Doris数据仓库平台的数据迁移过程。包含稳定性及迁移性能测试(JMeter)、负载平衡等。
  % \item 学习Kerberos的安全通信过程;Grafana服务器监视平台的搭建;数据仓库Web前端的Debug。
\end{itemize}

% \datedsubsection{\textbf{DID-ACTE} 荷兰莱顿}{2015年3月 - 2015年6月}
% \role{本科毕业设计}{LIACS 交换生}
% 利用结巴分词对中国古文进行分词与词性标注,用已有领域知识训练形成 classifier 并对结果进行调优
% \begin{onehalfspacing}
% \begin{itemize}
%   \item 利用结巴分词对中国古文进行分词与词性标注
%   \item 利用已有领域知识训练形成 classifier, 并用分词结果进行测试反馈
%   \item 尝试不同规则,对 classifier 进行调优
% \end{itemize}
% \end{onehalfspacing}

\section{Paper \& Project}
\datedsubsection{\textbf{Project Paper}, DEIM 2020}{2019.07 - 2020.03}
\role{The collaboration of the Real Sakai Lab, Waseda University and Wider Planet, Inc.}{}
《Purchase Prediction based on Recurrent Neural Networks with an Emphasis on Recent User Activities》
\begin{itemize}
  \item Use the BigTable of the Google Cloud Platform to analyze and process the Yoochoose dataset.
  \item Suggest three kinds of user activity aggregating methods to predict user purchase activity in the Yoochoose dataset and evaluate the performance. 
  % \item 利用Google Cloud Platform的BigTable进行Yoochoose数据集的处理及特征工程。
  % \item 提出三种用户行为合并算法并检测其性能,在Yoochoose数据集上进行用户购买行为预测实验。
\end{itemize}

\datedsubsection{\textbf{Graduation Project}}{2018.01 - 2018.06}
《Audio Recognition and Noise-Canceling based on RNN》
\begin{itemize}
  \item Build a 3-layer-GRU model to learn human voice features, extract clear human voices from noisy environments.
  \item Build a Chinese Voice-Text dataset by drama and subtitles, including 100GB with 370000 wave files.
  % \item 主要:构建3层GRU人声识别模型,过滤其他噪声,以输出清晰的人声,达到降噪目的。
  % \item 附加:通过处理中文电视剧及其字幕,构造了100GB、包含约370000件的中文音频-文字数据集。
\end{itemize}

\datedsubsection{\textbf{TIIC National Undergraduate IoT Design Contest}}{2017.04 - 2017.09}
《Intelligent Pram》
\begin{itemize}
  \item A pram model with sound location and auto obstacle avoidance features using 4 pieces of CC3200 boards.
  \item Develop the supplementary Android application independently to control the pram's movement and features.
  % \item 利用4块CC3200并构建局域网,完成了一个具有声音定位及自动避障移动等功能的婴儿车模型。
  % \item 独自开发了配套的Android应用,以实现对婴儿车的移动控制及相关功能的使用。
\end{itemize}


\section{Contest and Awards}
% increase linespacing [parsep=0.5ex]
\begin{itemize}
%   \item LeetCodeOJ Solutions, \textit{https://github.com/hijiangtao/LeetCodeOJ}
  \item TIIC National Undergraduate IoT Design Contest, \textbf{National First Prize} \& \textbf{Best Popularity Prize}.
  \item Programming Contests in Beijing Institute of Technology, \textbf{Third Prize} for several times.
  \item Scholarship in Beijing Institute of Technology, \textbf{Third Prize} for several times.
  % \item 2017年全国大学生物联网设计竞赛,\textbf{全国一等奖、最佳人气奖}
  % \item 北京理工大学程序设计竞赛,\textbf{三等奖若干次}
  % \item 北京理工大学,\textbf{三等奖学金若干次}
%   \item 电视节目"爸爸去哪儿"可视化分析展示, \textit{https://hijiangtao.github.io/variety-show-hot-spot-vis/}
\end{itemize}

% \section{\faHeartO\ 项目/作品摘要}
% \section{项目/作品摘要}
% \datedline{\textit{An Integrated Version of Security Monitor Vis System}, https://hijiangtao.github.io/ss-vis-component/ }{}
% \datedline{\textit{Dark-Tech}, https://github.com/hijiangtao/dark-tech/ }{}
% \datedline{\textit{融合社交网络数据挖掘的电视节目可视化分析系统}, https://hijiangtao.github.io/variety-show-hot-spot-vis/}{}
% \datedline{\textit{LeetCodeOJ Solutions}, https://github.com/hijiangtao/LeetCodeOJ}{}
% \datedline{\textit{Info-Vis}, https://github.com/ISCAS-VIS/infovis-ucas}{}


% \section{\faInfo\ 社会实践/其他}
\section{Activities}
% increase linespacing [parsep=0.5ex]
\begin{itemize}
  \item Server administrator and lab safety officer, Real Sakai Lab, Waseda University.
  \item Class Vice-Monitor (4 years), Minister in the Student Union (1 year), Beijing Institute of Technology.
  \item TOEFL iBT: 80; TOEIC: 750; JLPT N2: 161; Putonghua Proficiency Test: Level 1-B.
  \item Road bike, Randonneurs 200KM; Accordion, Amateur Grade 8th.
  \item GitHub: https://github.com/ShinoharaYuuyoru
  \item LinkedIn: https://www.linkedin.com/in/quanyu-piao/
  \item Blog: https://shinoharayuuyoru.github.io/
  % \item 早稻田大学Real Sakai Lab,服务器管理员、安全员。
  % \item 北京理工大学,07111403班副班长(4年);计算机学院学生会文体部部长(1年)。
  % \item TOEFL iBT:80;TOEIC:750;日本语能力测试(JLPT)N2:161;普通话水平测试:一级乙等
  % \item 公路自行车,不间断骑行200KM挑战证书;手风琴,业余8级。
  % \item GitHub: https://github.com/ShinoharaYuuyoru
  % \item LinkedIn: https://www.linkedin.com/in/quanyu-piao/
  % \item 博客:https://shinoharayuuyoru.github.io/
\end{itemize}

%% Reference
%\newpage
%\bibliographystyle{IEEETran}
%\bibliography{mycite}
\end{document}
