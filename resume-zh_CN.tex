% !TEX TS-program = xelatex
% !TEX encoding = UTF-8 Unicode
% !Mode:: "TeX:UTF-8"

\documentclass{resume}
\usepackage{zh_CN-Adobefonts_external} % Simplified Chinese Support using external fonts (./fonts/zh_CN-Adobe/)
%\usepackage{zh_CN-Adobefonts_internal} % Simplified Chinese Support using system fonts
\usepackage{linespacing_fix} % disable extra space before next section
\usepackage{cite}

\begin{document}
\pagenumbering{gobble} % suppress displaying page number

\name{朴泉宇}

% {E-mail}{mobilephone}{homepage}
% be careful of _ in emaill address
\contactInfo{(+86)15624951831}{quanyu.piao@gmail.com}{}{Web后端研发}
% {E-mail}{mobilephone}
% keep the last empty braces!
%\contactInfo{xxx@yuanbin.me}{(+86) 131-221-87xxx}{}
 
% \section{个人总结}
% 本人在校成绩优秀、乐观向上,工作负责、自我驱动力强、热爱尝试新事物,认同开放、连接、共享的Web在未来的不可替代性。在校期间长期从事可视分析(Web的2D/3D时空可视化)相关研究,对Web技术发展趋势及前端工程化解决方案有浓厚兴趣。\textbf{现任职于阿里巴巴集团。}

% \section{\faGraduationCap\ 教育背景}
\section{教育背景}
\datedsubsection{\textbf{早稻田大学},计算机与通信技术,\textit{硕士(在读)}}{2019.04 - 2021.03}
\datedsubsection{\textbf{北京理工大学},计算机科学与技术,\textit{工学学士}}{2011.09 - 2014.06}
% \section{\faCogs\ IT 技能}
\section{技术能力}
% increase linespacing [parsep=0.5ex]
\begin{itemize}
  \item \textbf{编程语言}: C, C++, Python, Java(Android), Swift(iOS)
  \item \textbf{框架}: Doris(Apache); PyTorch, TensorFlow 1.x; OpenCV
  \item \textbf{关键词}: 数据仓库 / 机器学习 / 客户端
\end{itemize}

% \end{itemize}

\section{实习经历}
\datedsubsection{\textbf{SOFTBRAIN株式会社}, iOS开发工程师}{2019.08 - 2020.05}
\begin{itemize}
  \item 小组(2人)负责开发并优化iOS客户端名片识别模块(OpenCV)。
  \item 独立负责一部分原有的Objective-C项目向Swift的迁移。
\end{itemize}

\datedsubsection{\textbf{Amazon Japan}, 短期实习}{2019.09}
\begin{itemize}
  \item 体验Amazon Alexa Skills的开发。
\end{itemize}

\datedsubsection{\textbf{搜狐}, 数据仓库工程师}{2018.12 - 2019.03}
\begin{itemize}
  \item 参与向Doris数据仓库平台的数据迁移过程。包含稳定性及迁移性能测试(JMeter)、负载平衡等。
  \item 学习Kerberos的安全通信过程;Grafana服务器监视平台的搭建;数据仓库Web前端的Debug。
\end{itemize}

% \datedsubsection{\textbf{DID-ACTE} 荷兰莱顿}{2015年3月 - 2015年6月}
% \role{本科毕业设计}{LIACS 交换生}
% 利用结巴分词对中国古文进行分词与词性标注,用已有领域知识训练形成 classifier 并对结果进行调优
% \begin{onehalfspacing}
% \begin{itemize}
%   \item 利用结巴分词对中国古文进行分词与词性标注
%   \item 利用已有领域知识训练形成 classifier, 并用分词结果进行测试反馈
%   \item 尝试不同规则,对 classifier 进行调优
% \end{itemize}
% \end{onehalfspacing}

\section{论文及项目}
\datedsubsection{\textbf{参与项目并发表论文}, DEIM 2020}{2019.07 - 2020.03}
\role{早稻田大学Real Sakai Lab与Wider Planet, Inc.的合作项目}{}
《Purchase Prediction based on Recurrent Neural Networks with an Emphasis on Recent User Activities》
\begin{itemize}
  \item 利用Google Cloud Platform的BigTable进行Yoochoose数据集的处理及特征工程。
  \item 提出三种用户行为合并算法并检测其性能,在Yoochoose数据集上进行用户购买行为预测实验。
\end{itemize}

\datedsubsection{\textbf{本科毕业设计项目}}{2018.01 - 2018.06}
《手机通话的自动降噪系统(Audio Recognition and Noise-Canceling based on RNN)》
\begin{itemize}
  \item 主要:构建3层GRU人声识别模型,过滤其他噪声,以输出清晰的人声,达到降噪目的。
  \item 附加:通过处理中文电视剧及其字幕,构造了100GB、包含约370000件的中文音频-文字数据集。
\end{itemize}

\datedsubsection{\textbf{2017年全国大学生物联网设计竞赛}}{2017.04 - 2017.09}
《基于声音定位的智能婴儿车(Intelligent Pram)》
\begin{itemize}
  \item 利用4块CC3200并构建局域网,完成了一个具有声音定位及自动避障移动等功能的婴儿车模型。
  \item 独自开发了配套的Android应用,以实现对婴儿车的移动控制及相关功能的使用。
\end{itemize}


\section{竞赛及获奖}
% increase linespacing [parsep=0.5ex]
\begin{itemize}
%   \item LeetCodeOJ Solutions, \textit{https://github.com/hijiangtao/LeetCodeOJ}
  \item 2017年全国大学生物联网设计竞赛,\textbf{全国一等奖、最佳人气奖}
  \item 北京理工大学程序设计竞赛,\textbf{三等奖若干次}
  \item 北京理工大学,\textbf{三等奖学金若干次}
%   \item 电视节目"爸爸去哪儿"可视化分析展示, \textit{https://hijiangtao.github.io/variety-show-hot-spot-vis/}
\end{itemize}

% \section{\faHeartO\ 项目/作品摘要}
% \section{项目/作品摘要}
% \datedline{\textit{An Integrated Version of Security Monitor Vis System}, https://hijiangtao.github.io/ss-vis-component/ }{}
% \datedline{\textit{Dark-Tech}, https://github.com/hijiangtao/dark-tech/ }{}
% \datedline{\textit{融合社交网络数据挖掘的电视节目可视化分析系统}, https://hijiangtao.github.io/variety-show-hot-spot-vis/}{}
% \datedline{\textit{LeetCodeOJ Solutions}, https://github.com/hijiangtao/LeetCodeOJ}{}
% \datedline{\textit{Info-Vis}, https://github.com/ISCAS-VIS/infovis-ucas}{}


% \section{\faInfo\ 社会实践/其他}
\section{课余活动及其他}
% increase linespacing [parsep=0.5ex]
\begin{itemize}
  \item 早稻田大学Real Sakai Lab,服务器管理员、安全员。
  \item 北京理工大学,07111403班副班长(4年);计算机学院学生会文体部部长(1年)。
  \item TOEFL iBT:80;TOEIC:750;日本语能力测试(JLPT)N2:161;普通话水平测试:一级乙等
  \item 公路自行车,不间断骑行200KM挑战证书;手风琴,业余8级。
  \item GitHub: https://github.com/ShinoharaYuuyoru
  \item LinkedIn: https://www.linkedin.com/in/quanyu-piao/
  \item 博客:https://shinoharayuuyoru.github.io/
\end{itemize}

%% Reference
%\newpage
%\bibliographystyle{IEEETran}
%\bibliography{mycite}
\end{document}
